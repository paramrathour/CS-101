\usepackage[utf8]{inputenc}
\usepackage[DIV=16]{typearea}
\usepackage{titlesec}
\titleformat{\section}[block]{\sffamily\Large\filcenter\bfseries}{\S\thesection.}{0.25cm}{\Large}
\titleformat{\subsection}[block]{\vspace{-0.7em}\large\bfseries\sffamily}{\thesubsection.}{0.2cm}{\large}
% \newcommand{\sectionbreak}{\clearpage}
% \newcommand{\subsectionbreak}{\clearpage}
% \newcommand\sectionbreak{\ifnum\value{section}>1\clearpage\fi}
\newcommand\subsectionbreak{\ifnum\value{subsection}>1\clearpage\fi}
\usepackage[parfill]{parskip}
\setlength{\parindent}{0em}
\usepackage{enumitem}
\usepackage[linesnumbered,ruled]{algorithm2e}

\usepackage{amsmath, amssymb, amsfonts, amsthm, mathtools}
\usepackage{nccmath, bm}
\newtheorem{theorem}{Theorem}
\newtheorem*{note}{Note}
\newtheorem*{funvideo}{Fun Video}
% \newtheorem*{example}{Example}
\newtheorem*{topics}{Topics}
\newtheorem*{noteI}{Interesting Observation}
\newtheorem*{fact}{Fun Fact}
\newtheorem*{hint}{Hint}
\theoremstyle{remark}
\newtheorem*{sol}{Solution}

\usepackage[dvipsnames]{xcolor}
\usepackage{tikz}
\usetikzlibrary{shapes.geometric}
\usetikzlibrary{calc}
\newcommand\encircle[1]{%
  \tikz[baseline=(X.base)] 
    \node (X) [draw, shape=circle, inner sep=0] {\strut #1};}

\usepackage{anyfontsize}
% \usepackage{emoji}
\usepackage[most]{tcolorbox}
% \usepackage{sidecap}
\usepackage{graphicx}
%\usepackage{minipage}
\usepackage{subcaption}
\usepackage{wrapfig}
\usepackage{float}
% \usepackage{minted}

\usepackage{skak}
\usepackage{datetime2}
\usepackage{relsize}
\usepackage{url}
\usepackage{cprotect}
\usepackage{comment}
\usepackage{lipsum}
\usepackage{epigraph}

\usepackage[colorlinks=true]{hyperref}
\hypersetup{
	linktoc= all,     %set to all if you want both sections and subsections linked
	urlcolor= blue,
	pdftitle={Practice Problems},
	pdfauthor = {Param Rathour, Department of Electrical Engineering, Indian Institute of Technology Bombay},
	pdfsubject={CS 101 Computer Programming and Utilization},
}