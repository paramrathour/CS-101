\section{Iteration Domination}
\begin{topics}
\verb!for, while & do while! loops and previous sections.
\end{topics}
\subsection{Pisano Period}
The Fibonacci numbers are the numbers in the integer sequence defined by the following recurrence relation
\begin{equation}
	\begin{aligned}
		F_0 &= 0\\
		F_1 &= 1 \\
		F_n &= F_{n-1} + F_{n-2}\quad n \in \mathbb{Z}\quad\text{(Yes! They can be extended to negative numbers)}
	\end{aligned}
\end{equation}
For any integer $n$, the sequence of Fibonacci numbers $F_i \ \%\ n$ is periodic.

The Pisano period, denoted $\pi(n)$, is the length of the period of this sequence.

For example, the sequence of Fibonacci numbers modulo 3 begins:
\begin{equation*}
	0, 1, 1, 2, 0, 2, 2, 1, 0, 1, 1, 2, 0, 2, 2, 1, 0, 1, 1, 2, 0, 2, 2, 1, 0,\ldots\text{(\href{https://oeis.org/A082115}{A082115})}
\end{equation*}
This sequence has period 8, so $\pi(3) = 8$.

Basically, the remainders repeat when these numbers are divided by $n$. You have to find this period.

\textbf{Problem Statement:}\\
Find Pisano period of $t$ numbers $n_1,n_2,\ldots,n_t$
\begin{testcases}
	{$t$ \hfill(number of test cases, an integer)\\
	$n_1\ n_2\ \ldots\ n_t$ \hfill($t$ space seperated numbers for each testcase)}
	{$\pi(n_i)$ \hfill(each test case space seperated)}
	{$1 < n_i \leq 1000$}
	{17\\2 3 5 8 13 21 34 55 89 144 233 987 30 50 98 750 1000}
	{3\quad8\quad20\quad12\quad28\quad16\quad36\quad20\quad44\quad24\quad52\quad32\quad120\quad300\quad336\quad3000\quad1500}
	{https://github.com/paramrathour/CS-101/tree/main/Starter Codes/Pisano Period.cpp}
\end{testcases}
\begin{funvideo}
\href{https://youtu.be/Nu-lW-Ifyec}{Fibonacci Mystery -- Numberphile}
\end{funvideo}
\subsection{Palindromic Number}
A non-negative integer is a Palindromic number if it remains the same when it's digits are reversed.

\textbf{Problem Statement:}\\
Determine whether the given integer is a Palindrome for all test cases.
\begin{testcases}
	{$t$ \hfill(number of test cases, an integer)\\
	$n_1\ n_2\ \ldots\ n_t$ \hfill($t$ space seperated integers for each testcase)}
	{``yes'' if $n_i$ is a Palindrome else ``no''. \hfill(each test case on a newline)}
	{$0 \leq n_i \leq 10^{9}$}
	{13\\1 7 15 22 196 666 1212 96096 111111 8801088 9256713 40040004 123454321}
	{yes\\yes\\no\\yes\\no\\yes\\no\\no\\yes\\yes\\no\\no\\yes}
	{https://github.com/paramrathour/CS-101/tree/main/Starter Codes/Palindromic Number.cpp}
\end{testcases}
\begin{funvideo}
\href{https://youtu.be/4fE_sXZjxng}{Why 02/02/2020 is the most palindromic date ever. -- Stand-up Maths}\\
\href{https://youtu.be/OKhacWQ2fCs}{Every Number is the Sum of Three Palindromes -- Numberphile}
\end{funvideo}
\subsection{Kempner Series}
Kempner series is \hyperref[pp:harmonic]{Harmonic} series where all terms whose denominator contains 9 are excluded.
\begin{equation}
{K_{n}=1+{\frac {1}{2}}+{\frac {1}{3}}+\cdots+{\frac {1}{8}}+{\frac {1}{10}}+\cdots +{\frac {1}{n}}=\sum _{i=1}^{n}c_i{\frac {1}{i}}} \ \text{where } c_i = \begin{cases} 
      0 & \text{if $i$'s decimal expansion contains a }9\\
      1 & \text{else}
   \end{cases}
\end{equation}
\begin{fact}
Unlike Harmonic series, the Kempner series \textbf{converges} to around 22.92.\\This is because most large integers contain a 9, hence they will be excluded from the sum.
\end{fact}
\textbf{Problem Statement:}\\
Calculate $K_n$ for all test cases accurate till 10 decimal places.
\begin{testcases}
	{$t$ \hfill(number of test cases, an integer)\\
	$n_1\ n_2\ \ldots\ n_t$ \hfill($t$ space seperated integers for each testcase)}
	{$K_{n_i}$ \hfill(each test case on a newline, accurate till 10 decimal places)}
	{$1 \leq n_i \leq 10^{6}$}
	{11\\1 2 3 5 10 20 30 50 100 1000 1000000}
	{1.0000000000\\1.5000000000\\1.8333333333\\2.2833333333\\2.8178571429\\3.4339969671\\3.7967616822\\4.2549307007\\4.7818487651\\6.5907201903\\11.0156518499}
	{https://github.com/paramrathour/CS-101/tree/main/Starter Codes/Kempner Series.cpp}
\end{testcases}
\begin{funvideo}
\href{https://youtu.be/UfEiJJGv4CE}{3 is everywhere -- Numberphile}
\end{funvideo}
\subsection{Base --2}
By using $-2$ as the base, both positive and negative integers can be expressed without an explicit sign or other irregularity. Just like positive integral bases, any base $-2$ number can be represented as follows:
\begin{equation}
(a_n\ldots a_2a_1a_0)_{(-2)} = a_n(-2)^n+\cdots+a_2(-2)^2+a_1(-2)^1+a_0(-2)^0 \quad\text{where $a_i$ is either 0 or 1}
\end{equation}
To find base $-2$ representation of $n$, we repeatedly divide by $-2$ until the quotient becomes 0 and the remainders generated (which are either 0 or 1) will be the digits of base $-2$ representation.
\begin{equation*}
n = \text{Quotient}\times(-2) + \text{Reminder} \quad\rightarrow\quad \text{Quotient} = \text{Quotient}_{\text{new}}\times(-2) + \text{Reminder}_{\text{new}}
\end{equation*}
For $-3$, the process it as shown below,
\begin{equation*}
\begin{aligned}
-3&= 2\times(-2) + {1} &\quad\rightarrow\quad a_0 = 1\\
2&= -1\times(-2) + {0} &\quad\rightarrow\quad a_1 = 0\\
-1&= 1\times(-2) + {1} &\quad\rightarrow\quad a_2 = 1\\
1&= 0\times(-2) + {1} &\quad\rightarrow\quad a_3 = 1
\end{aligned}
\end{equation*}
Hence $(-3)_{10} = (1101)_{(-2)}$.
\begin{note}
C++'s \% operator may give negative values when the dividend or divisor is negative.\\
For example, $(-1)\%(2) = (-1)\%(-2) = -1 \neq 1$.
\end{note}

\textbf{Problem Statement:}\\
Convert the given decimal number into base $-2$ for all test cases.
\begin{testcasesMore}
	{$t$ \hfill(number of test cases, an integer)\\
	$n_1\ n_2\ \ldots\ n_t$ \hfill($t$ space seperated integers for each testcase)}
	{Converted base $-2$ number \hfill(each test case on a newline)}
	{$-200 \leq n_i \leq 200$}
	{10\\-4 -3 -2 -1 0 1 2 3 4 100}
	{1100\\1101\\10\\11\\0\\1\\110\\111\\100\\110100100}
	{https://github.com/paramrathour/CS-101/tree/main/Test Cases/Base -2/Input.txt}
	{https://github.com/paramrathour/CS-101/tree/main/Test Cases/Base -2/Output.txt}
	{https://github.com/paramrathour/CS-101/tree/main/Starter Codes/Base -2.cpp}
\end{testcasesMore}
\begin{funvideo}
\href{https://youtu.be/U6xJfP7-HCc}{Base 12 -- Numberphile}
\end{funvideo}
\subsection{Base Conversion}{\label{pp:baseconversion}}
In this problem, you will convert binary number to decimal and vice versa.
\begin{hint}
First solve the conversion problem for integers and then try to incorporate their fractional part.
\end{hint}
\begin{enumerate}[label=(\alph*)]
\item 
\textbf{Problem Statement:}\\
Convert $t$ positive binary numbers ($n_1,n_2,\ldots,n_t$) to decimal.
\begin{testcases}
	{$t$ \hfill(number of test cases, an integer)\\
	$n_1\ n_2\ \ldots\ n_t$ \hfill($t$ space seperated numbers for each testcase)}
	{Converted decimal number \hfill(space seperated)}
	{$0\leq n_i\leq 10^{15}$, a maximum of 8 digits after binary point (`.') \hfill(base 2, a double)}
	{9\\1 111 110001 101010111 100101100001 1.00011001 11.001001 110.01 10110.01110101}
	{1.00000000\quad7.00000000\quad49.00000000\quad343.00000000\quad2401.00000000\quad1.09765625\quad3.14062500\quad6.25000000\quad22.45703125}
	{https://github.com/paramrathour/CS-101/tree/main/Starter Codes/Base Conversion I.cpp}
\end{testcases}
\item 
\textbf{Problem Statement:}\\
Convert $t$ positive decimal numbers ($n_1,n_2,\ldots,n_t$) to binary.
\begin{testcases}
	{$t$ \hfill(number of test cases, an integer)\\
	$n_1\ n_2\ \ldots\ n_t$ \hfill($t$ space seperated numbers for each testcase)}
	{Converted binary number truncated till 8 decimal places \hfill(space seperated)}
	{$0\leq n_i\leq 2500$, a maximum of 8 digits after decimal point (`.') \hfill(base 10, a double)}
	{9\\1 7 49 343 2401 1.1 3.1415 6.25 22.459}
	{1.00000000\quad111.00000000\quad110001.00000000\quad101010111.00000000\quad100101100001.00000000\quad1.00011001\quad11.00100100\quad110.01000000\quad10110.01110101}
	{https://github.com/paramrathour/CS-101/tree/main/Starter Codes/Base Conversion II.cpp}
\end{testcases}
\end{enumerate}
\begin{funvideo}
\href{https://youtu.be/xNx3JxRhnZE}{Dungeon Numbers -- Numberphile}
\end{funvideo}