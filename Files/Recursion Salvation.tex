\section{Recursion Salvation}
\begin{topics}
\verb!recursive functions! and previous sections.
\end{topics}
\paragraph{5 Simple Steps for Solving Any Recursive Problem}
(Courtesy -- \href{https://youtu.be/ngCos392W4w}{Reducible})
\begin{itemize}
	\item What's the simplest possible input?
	\item Play around with examples and visualize!
	\item Relate hard cases to simpler cases
	\item Generalize the pattern
	\item Write code by combining recursive pattern with base case
\end{itemize}
\subsection{Ackermann Function}
Ackermann Function is defined as follows
\begin{equation}
	\begin{aligned}
		\op{A}(0,n)&=n+1\\
		\op{A}(m,0)&=\op{A}(m-1,1)\\
		\op{A}(m,n)&=\op{A}(m-1,\op{A}(m,n-1))
	\end{aligned}
\end{equation}
\textbf{Problem Statement:}\\
Calculate $\op{A}(m,n)$ (given $m,n$) for all test cases.
\begin{testcasesFunction}
	{$t$ \hfill(number of test cases, an integer)\\
	$m_1\ n_1\ \quad m_2\ n_2\ \quad \ldots\quad m_t\ n_t$ \hfill($t$ space seperated integer pairs for each testcase)}
	{$\op{A}(m_i,n_i)$\hfill(each on a newline)}
	{$m_i,n_i$ are postive integers such that $\op{A}(m_i,n_i)$ is within the range of int}
	{\texttt{int A(int m, int n)} -- returns $A(m,n)$}
	{10\\0 0\quad0 5\quad1 0\quad1 3\quad2 4\quad3 1\quad3 3\quad3 9\quad4 0\quad4 1}
	{1\\6\\2\\5\\11\\13\\61\\4093\\13\\65533}
	{https://github.com/paramrathour/CS-101/tree/main/Starter Codes/Ackermann Function.cpp}
\end{testcasesFunction}
\begin{noteI}
	Was your program able to compute the last output? Why not? How to fix it?
\end{noteI}
\begin{funvideo}
\href{https://youtu.be/i7sm9dzFtEI}{The Most Difficult Program to Compute? -- Computerphile}
\end{funvideo}
\subsection{Horner's Method}
Consider, the problem of evaluating a polynomial given its coefficients.
\begin{equation*}
	f(x)=a_{0}+a_{1}\cdot x+a_{2}\cdot x^{2}+a_{3}\cdot x^{3}+\cdots +a_{n}\cdot x^{n}
\end{equation*}
A naive method is to evaluate $x^0, x^1, x^2, \ldots, x^n$ independently, then multiply $x^i$ with $a_i$ and add all results.
\begin{equation*}
	f(x)=a_{0}+a_{1}\cdot x+a_{2}\cdot x\cdot x+a_{3}\cdot x\cdot x\cdot x+\cdots +a_{n}\underbrace{x\cdot x\cdots x}_{n\text{ times}}
\end{equation*}
This approach takes $1+2+\cdots+n=n(n+1)/2$ multiplications and $n$ additions.\\
It can be improved by using the precalculated $x^{i-1}$ and multiplying it by $x$ to get $x^{i}$. This reduces the number of multiplications significantly to $2n-1$ while keeping the number of additions $n$.
\begin{equation*}
	f(x)=a_{0}+a_{1}\cdot x^0\cdot x+a_{2}\cdot x^1\cdot x+a_{3}\cdot x^2\cdot x+\cdots +a_{n}x^{n-1}\cdot x
\end{equation*}
But surprisingly there is an even better way! Horner's Method as described in \ref{eq:horner}, is an optimal algorithm for polynomial evaluation needing only $n$ multiplications and $n$ additions.
\begin{equation}{\label{eq:horner}}
	f(x)=a_{0}+x\bigg(a_{1}+x\Big(a_{2}+x\big(a_{3}+\cdots +x(a_{n-1}+x\,a_{n})\cdots \big)\Big)\bigg)
\end{equation}
\textbf{Problem Statement:}\\
Evaluate polynomial given by coefficients at $x$ using Horner's Method for all test cases.
\begin{testcasesFunction}
	{$t$ \hfill(number of test cases, an integer)\\
	$x_i\qquad n_i \quad a_0\ a_1\ a_2 \cdots a_n$ \hfill($n_i+3$ space seperated integers for each testcase)}
	{$f(x_i)$\hfill(each on a newline)}
	{$1 < x_i,\ n_i, a_i \leq 10^{4}$\\
	Also assume that the calculations are always within the range of long long}
	{\texttt{long long f(const int \&x, int a, int b)} -- returns $f(x)$, you are also given two extra parameters.}
	{6\\1\qquad 0\quad 1\\2\qquad 1\quad -3 2\\2\qquad 2\quad 15 -8 7\\3\qquad 3\quad 2 -1 -3 4\\5\qquad 6\quad 21 10 19 47 48 9 27\\3\qquad 14\quad -1 59 265 -35 8 -97 -932 38 4 -62 -643 38 -3 27 950}
	{1\\1\\27\\80\\486421\\4552224296}
	{https://github.com/paramrathour/CS-101/tree/main/Starter Codes/Horner's Method.cpp}
\end{testcasesFunction}
\begin{funvideo}
\href{https://youtu.be/cUzklzVXJwo}{How Imaginary Numbers Were Invented -- Veritasium}
\end{funvideo}
\subsection{Partitions}
A partition of a natural number $n$ is a way of decomposing $n$ as sum of natural numbers $\leq n$.\\
For example, their are 5 partitions of 4 given by $\{4, 3+1, 2+2, 2+1+1, 1+1+1+1\}$.\\
Let use denote the number of partitions of $n$ by $\op{P}(n)$.\\
Now, we move to a seemingly unrelated theorem.
\begin{theorem}[Pentagonal Number Theorem]
	PNT relates the product and series representations of the \href{https://en.wikipedia.org/wiki
	/Euler_function}{Euler function}
	\begin{equation}{\label{eq:pnt}}
		\prod_{n=1}^{\infty}\left(1-x^{n}\right)=\sum_{k=-\infty }^{\infty }\left(-1\right)^{k}x^{k\left(3k-1\right)/2}=1+\sum _{k=1}^{\infty }(-1)^{k}\left(x^{k(3k+1)/2}+x^{k(3k-1)/2}\right)
	\end{equation}
	In other words,
	\begin{equation*}
		(1-x)(1-x^{2})(1-x^{3})\cdots =1-x-x^{2}+x^{5}+x^{7}-x^{12}-x^{15}+x^{22}+x^{26}-\cdots
	\end{equation*}
	The exponents $1, 2, 5, 7, 12,\ldots$ on the right hand side are called (generalized) pentagonal numbers (\href{https://oeis.org/A001318}{A001318}).\\
	They are given by the formula $p_k = k(3k - 1)/2$ for $k = 1, -1, 2, -2, 3,-3,\ldots$
\end{theorem}
Equation \ref{eq:pnt} implies a recurrence relation for calculating $\op{P}(n)$ given by
\begin{equation}
	\op{P}(n)=\op{P}(n-1)+\op{P}(n-2)-\op{P}(n-5)-\op{P}(n-7)+\cdots = \sum_{k\neq 0}(-1)^{k-1}\op{P}(n-p_{k})
\end{equation}
\textbf{Problem Statement:}\\
Calculate $\op{P}(n)$ for all test cases using\ref{eq:pnt} or otherwise :).
\begin{testcasesFunction}
	{$t$ \hfill(number of test cases, an integer)\\
	$n_1\ n_2\ \ldots\ n_t$ \hfill($t$ space seperated integers for each testcase)}
	{$\op{P}(n_i)$ \hfill(each test case on a newline)}
	{$1 \leq n_i \leq 40$}
	{\texttt{int P(int n)} -- returns $\op{P}(n)$ }%\hfill(try solving with and without using the given extra parameter $k$ :D)}
	{9\\1 2 3 4 5 10 20 30 40}
	{1\\2\\3\\5\\7\\42\\627\\5604\\37338}
	{https://github.com/paramrathour/CS-101/tree/main/Starter Codes/Partitions.cpp}
\end{testcasesFunction}
\begin{noteI}
If the last output took a long time then think how you can do the calculations faster?
\end{noteI}
\begin{funvideo}
\href{https://youtu.be/NjCIq58rZ8I}{Partitions -- Numberphile}\\
\href{https://youtu.be/iJ8pnCO0nTY}{The hardest What comes next (Euler's pentagonal formula) -- Mathologer}
\end{funvideo}
\KOMAoptions{paper=A3}
\recalctypearea
\subsection{Modular Exponentiation}
Consider the problem of calculating $x^y\Mod k$ (i.e. the remainder when $x^y$ is divided by $k$).

A naive approach is to keep multiplying by $x$ (and take$\Mod k$) until we reach $x^y$.\footnote{this works because $(a\cdot b)\Mod m =  ((a\Mod m)\cdot(b\Mod m))\Mod m$}
\begin{equation*}
x \Mod k \rightarrow x^2 \Mod k \rightarrow x^3 \Mod k \rightarrow x^4 \Mod k \rightarrow \cdots \rightarrow x^y \Mod k
\end{equation*}
We can use a much faster method which involves \emph{repeated squaring} of $x \Mod k$
\begin{equation}
x \Mod k \rightarrow x^2 \Mod k \rightarrow x^4 \Mod k \rightarrow x^8 \Mod k \rightarrow \cdots \rightarrow x^{2^{\lfloor\log y\rfloor}} \Mod k
\end{equation}
The idea is to multiply some of the above numbers and get $x^y \Mod k$.\\
This is achieved by choosing all powers that have 1 in binary representation of $y$.\\
For example,
\begin{equation*}
x^{25} = x ^ {11001_2} = x ^ {10000_2} \cdot x ^ {1000_2} \cdot x ^ {1_2} = x ^ {16} \cdot x ^ 8 \cdot x ^ 1
\end{equation*}
which gives,
\begin{equation*}
x^{25} \Mod k  = ((x ^ {16} \Mod k)\cdot (x ^ 8\Mod k)  \cdot (x ^ 1\Mod k)) \Mod k
\end{equation*}
\begin{enumerate}[label=(\alph*)]
\item 
\textbf{Problem Statement:}\\
Calculate $x^y\Mod k$ using the above method for $n$ $(x,y,k)$ triples. Take $k=10^9+7$. \href{https://www.geeksforgeeks.org/modulo-1097-1000000007/}{why this number?}
\begin{testcasesFunction}
	{$t$ \hfill(number of test cases, an integer)\\
	$x_1\ y_1\ \quad x_2\ y_2\ \quad \ldots\quad x_t\ y_t$ \hfill($t$ space seperated integer pairs for each testcase)}
	{$x_i^{y_i} \Mod k$ \hfill(each test case on a newline)}
	{$1 < x_i,\ y_i \leq 10^{9}$}
	{\texttt{int mod\_exp(int x, int y, int k)} -- returns $x^y \Mod k$}
	{5\\3 4\quad2 8\quad123 123\quad129612095 411099530\quad241615980 487174929}
	{81\\256\\921450052\\276067146\\838400234}
	{https://github.com/paramrathour/CS-101/tree/main/Starter Codes/Modular Exponentiation I.cpp}
\end{testcasesFunction}
\begin{note}
	Before proceeding to next task, verify your program on more testcases from \href{https://cses.fi/problemset/task/1095}{here}.
\end{note}
\item 
\textbf{Problem Statement:}\\
Calculate $x^y\Mod k$ using the above method for $n$ $(x,y,k)$ triples. Take $k=10^9+7$. \href{https://www.geeksforgeeks.org/modulo-1097-1000000007/}{why this number?}
\begin{testcasesFunction}
	{$t$ \hfill(number of test cases, an integer)\\
	$x_1\ y_1\ z_1\ \quad x_2\ y_2\ z_2\ \quad \ldots\quad x_t\ y_t z_t\ $ \hfill($t$ space seperated triples for each testcase)}
	{$x_i^{y_{i}^{z_i}} \Mod k$ \hfill(each test case on a newline)}
	{$1 < x_i,\ y_i,\ z_i \leq 10^{9}$}
	{\texttt{int mod\_super\_exp(int x, int y, int z, int k)} -- returns $x^{y^z} \Mod k$}
	{5\\3 7 1\quad15 2 2\quad3 4 5\quad427077162 725488735 969284582\quad690776228 346821890 923815306}
	{2187\\50625\\763327764\\464425025\\534369328}
	{https://github.com/paramrathour/CS-101/tree/main/Starter Codes/Modular Exponentiation II.cpp}
\end{testcasesFunction}
\begin{note}
	Verify your program on more testcases from \href{https://cses.fi/problemset/task/1712}{here}.
\end{note}
\end{enumerate}
\begin{funvideo}
\href{https://youtu.be/cbGB__V8MNk}{Square \& Multiply Algorithm - Computerphile}
\end{funvideo}
\subsection{Hereditary Representation}
The usual base $b$ representation is of a natural number is given by
\begin{equation}
 	n_b = a_0 \cdot b^0 + a_1 \cdot b^1 + \cdots \quad\text{where $a_i$'s $\in \{0,1,\ldots,b-1\}$}
\end{equation} Here the power $i$ of exponent $b^i$ is in decimal but what if we continue to represent $i$ in base $b$ until we use only $0, 1, 2, \ldots, b-1$ for all exponents of $b$.

This is the Hereditary Representation! Representing a natural number $n_b$ in base $b$ using only $0, 1, 2, \ldots, b-1$ as exponents of $b$.

To generate this representation, find the usual base representation of the number and then represent its exponents also in the usual base representation. Keep repeating this until there is no exponent $> b$.

For example, 
\begin{equation}
\begin{aligned}
666_2 &= 2^1 + 2^3 + 2^4 + 2^7 + 2^9\\
&= 2^1  + 2^{2^0+2^1} + 2^{2^2} + 2^{2^{0}+2^1+2^2} + 2^{2^{0}+2^3}\\
&= 2^{1} + 2^{2^{0} + 2^{1}} + 2^{2^{2^{1}}} + 2^{2^{0} + 2^{1} + 2^{2^{1}}} + 2^{2^{0} + 2^{2^{0} + 2^{1}}}
\end{aligned}
\end{equation}
Here are some more examples to get familiar,
\begin{align*}
10_2 &= 2^{1} + 2^{2^{0} + 2^{1}}\\
100_2 &= 2^{2^{1}} + 2^{2^{0} + 2^{2^{1}}} + 2^{2^{1} + 2^{2^{1}}}\\
3435_3 &= 2\cdot3^{1} + 3^{3^{1}} + 2\cdot3^{2\cdot3^{0} + 3^{1}} + 3^{2\cdot3^{1}} + 3^{3^{0} + 2\cdot3^{1}}\\
% 1000000000000000001_{10} &= A^{0} + A^{8\cdot A^{0} + A^{1}}
754777787027_{10} &= 7\cdot A^{0} + 2\cdot A^{1} + 7\cdot A^{3} + 8\cdot A^{4} + 7\cdot A^{5} + 7\cdot A^{6} + 7\cdot A^{7} + 7\cdot A^{8} + 4\cdot A^{9} + 5\cdot A^{A^{1}} + 7\cdot A^{A^{0} + A^{1}}
% 1162849439785405935_{10} &= 5\cdot A^{0} + 3\cdot A^{1} + 9\cdot A^{2} + 5\cdot A^{3} + 4\cdot A^{5} + 5\cdot A^{6} + 8\cdot A^{7} + 7\cdot A^{8} + 9\cdot A^{9} + 3\cdot A^{A^{1}} + 4\cdot A^{A^{0} + A^{1}} + 9\cdot A^{2\cdot A^{0} + A^{1}} + 4\cdot A^{3\cdot A^{0} + A^{1}} + 8\cdot A^{4\cdot A^{0} + A^{1}} + 2\cdot A^{5\cdot A^{0} + A^{1}} + 6\cdot A^{6\cdot A^{0} + A^{1}} + A^{7\cdot A^{0} + A^{1}} + A^{8\cdot A^{0} + A^{1}}
\end{align*}
\textbf{Problem Statement:}\\
Output the Hereditary Representation of the input natural number $n$ in base $b$ $(\geq2)$ following the below conventions:

\begin{itemize}
\item Use \verb!+, *! to denote addition (add space between operands), multiplication (no space between operands) respectively and \verb!b^{y}! for $b^y$ where $y$ is some expression.
\item The powers of base representation are in increasing order (first $b^0$ then $b^1$ then $b^2$ and so on).
\item Powers are displayed only when their coefficients are $>0$ (non-zero).
\item Coefficients themselves are only displayed when they are $>1$.
\item The exponents between 0 and $b-1$ must not be simplified further. So, $b$ is represented as \verb!b^{1}! and not as \verb!b^{b^{0}}!.
\item For bases $>10$, use capital alphabets $(A,B,C,\ldots,Z)$ to denote $(10,11,12,\ldots,35)$ respectively.
\end{itemize}
\begin{testcasesFunctionMore}
	{$t$ \hfill(number of test cases, an integer)\\
	$n_1\ b_1\ \quad n_2\ b_2\ \quad \ldots\quad n_t\ b_t$ \hfill($t$ space seperated pairs (number, base) for each testcase)}
	{Hereditary Representation of $n_i$ in base $b_i$\hfill(each on a newline)}
	{$1 < n_i \leq 2 \cdot 10^{18}$\\
	$1 < b_i \leq 35$}
	{\texttt{void Hereditary (long long num, int base)} -- prints the required representation}
	{9\\2 2\quad 10 2\quad 100 2\quad 666 3\quad 3435 3\quad3816547290 4\quad3816547290 9\quad 3816547290 35\quad1162849439785405935 10}
	{\texttt{2\^{}\{1\}}\\[0.7em]
\texttt{2\^{}\{1\} + 2\^{}\{2\^{}\{0\} + 2\^{}\{1\}\}}\\[0.7em]
\texttt{2\^{}\{2\^{}\{1\}\} + 2\^{}\{2\^{}\{0\} + 2\^{}\{2\^{}\{1\}\}\} + 2\^{}\{2\^{}\{1\} + 2\^{}\{2\^{}\{1\}\}\}}\\[0.7em]
\texttt{2\textsuperscript{*}3\^{}\{2\} + 2\textsuperscript{*}3\^{}\{3\^{}\{0\} + 3\^{}\{1\}\} + 2\textsuperscript{*}3\^{}\{2\textsuperscript{*}3\^{}\{0\} + 3\^{}\{1\}\}}\\[0.7em]
\texttt{2\textsuperscript{*}3\^{}\{1\} + 3\^{}\{3\^{}\{1\}\} + 2\textsuperscript{*}3\^{}\{2\textsuperscript{*}3\^{}\{0\} + 3\^{}\{1\}\} + 3\^{}\{2\textsuperscript{*}3\^{}\{1\}\} + 3\^{}\{3\^{}\{0\} + 2\textsuperscript{*}3\^{}\{1\}\}}\\[0.7em]
\texttt{2\textsuperscript{*}4\^{}\{0\} + 2\textsuperscript{*}4\^{}\{1\} + 4\^{}\{2\} + 3\textsuperscript{*}4\^{}\{3\} + 3\textsuperscript{*}4\^{}\{4\^{}\{1\}\} + 2\textsuperscript{*}4\^{}\{2\textsuperscript{*}4\^{}\{0\} + 4\^{}\{1\}\} + 3\textsuperscript{*}4\^{}\{3\textsuperscript{*}4\^{}\{0\} + 4\^{}\{1\}\} + 3\textsuperscript{*}4\^{}\{2\textsuperscript{*}4\^{}\{1\}\} + 2\textsuperscript{*}4\^{}\{4\^{}\{0\} + 2\textsuperscript{*}4\^{}\{1\}\} + 3\textsuperscript{*}4\^{}\{2\textsuperscript{*}4\^{}\{0\} + 2\textsuperscript{*}4\^{}\{1\}\} + 4\^{}\{3\textsuperscript{*}4\^{}\{0\} + 2\textsuperscript{*}4\^{}\{1\}\} + 3\textsuperscript{*}4\^{}\{3\textsuperscript{*}4\^{}\{1\}\} + 2\textsuperscript{*}4\^{}\{2\textsuperscript{*}4\^{}\{0\} + 3\textsuperscript{*}4\^{}\{1\}\} + 3\textsuperscript{*}4\^{}\{3\textsuperscript{*}4\^{}\{0\} + 3\textsuperscript{*}4\^{}\{1\}\}}\\[0.7em]
\texttt{2\textsuperscript{*}8\^{}\{0\} + 3\textsuperscript{*}8\^{}\{1\} + 7\textsuperscript{*}8\^{}\{2\} + 8\^{}\{3\} + 6\textsuperscript{*}8\^{}\{4\} + 7\textsuperscript{*}8\^{}\{5\} + 6\textsuperscript{*}8\^{}\{6\} + 3\textsuperscript{*}8\^{}\{7\} + 3\textsuperscript{*}8\^{}\{8\^{}\{1\}\} + 4\textsuperscript{*}8\^{}\{8\^{}\{0\} + 8\^{}\{1\}\} + 3\textsuperscript{*}8\^{}\{2\textsuperscript{*}8\^{}\{0\} + 8\^{}\{1\}\}}\\[0.7em]
\texttt{5\textsuperscript{*}A\^{}\{0\} + 3\textsuperscript{*}A\^{}\{1\} + 9\textsuperscript{*}A\^{}\{2\} + 5\textsuperscript{*}A\^{}\{3\} + 4\textsuperscript{*}A\^{}\{5\} + 5\textsuperscript{*}A\^{}\{6\} + 8\textsuperscript{*}A\^{}\{7\} + 7\textsuperscript{*}A\^{}\{8\} + 9\textsuperscript{*}A\^{}\{9\} + 3\textsuperscript{*}A\^{}\{A\^{}\{1\}\} + 4\textsuperscript{*}A\^{}\{A\^{}\{0\} + A\^{}\{1\}\} + 9\textsuperscript{*}A\^{}\{2\textsuperscript{*}A\^{}\{0\} + A\^{}\{1\}\} + 4\textsuperscript{*}A\^{}\{3\textsuperscript{*}A\^{}\{0\} + A\^{}\{1\}\} + 8\textsuperscript{*}A\^{}\{4\textsuperscript{*}A\^{}\{0\} + A\^{}\{1\}\} + 2\textsuperscript{*}A\^{}\{5\textsuperscript{*}A\^{}\{0\} + A\^{}\{1\}\} + 6\textsuperscript{*}A\^{}\{6\textsuperscript{*}A\^{}\{0\} + A\^{}\{1\}\} + A\^{}\{7\textsuperscript{*}A\^{}\{0\} + A\^{}\{1\}\} + A\^{}\{8\textsuperscript{*}A\^{}\{0\} + A\^{}\{1\}\}}}
	{https://github.com/paramrathour/CS-101/tree/main/Test Cases/Hereditary Representation/Input.txt}
	{https://github.com/paramrathour/CS-101/tree/main/Test Cases/Hereditary Representation/Output.txt}
	{https://github.com/paramrathour/CS-101/tree/main/Starter Codes/Horner's Method.cpp}
\end{testcasesFunctionMore}
\begin{funvideo}
\href{https://youtu.be/uWwUpEY4c8o}{Kill the Mathematical Hydra -- PBS Infinite Series}\\
\href{https://youtu.be/oBOZ2WroiVY}{How Infinity Explains the Finite -- PBS Infinite Series}
\end{funvideo}
\KOMAoptions{paper=A4}
\recalctypearea