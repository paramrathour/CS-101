\documentclass[../../Problems]{subfiles}
\begin{document}
\KOMAoptions{paper=A3}
\recalctypearea
\subsection{Hereditary Representation}
The usual base $b$ representation is of a natural number is given by
\begin{equation}
 	n_b = a_0 \cdot b^0 + a_1 \cdot b^1 + \cdots \quad\text{where $a_i$'s $\in \{0,1,\ldots,b-1\}$}
\end{equation} Here the power $i$ of exponent $b^i$ is in decimal but what if we continue to represent $i$ in base $b$ until we use only $0, 1, 2, \ldots, b-1$ for all exponents of $b$.

This is the Hereditary Representation! Representing a natural number $n_b$ in base $b$ using only $0, 1, 2, \ldots, b-1$ as exponents of $b$.

To generate this representation, find the usual base representation of the number and then represent its exponents also in the usual base representation. Keep repeating this until there is no exponent $> b$.

For example, 
\begin{equation}
\begin{aligned}
666_2 &= 2^1 + 2^3 + 2^4 + 2^7 + 2^9\\
&= 2^1  + 2^{2^0+2^1} + 2^{2^2} + 2^{2^{0}+2^1+2^2} + 2^{2^{0}+2^3}\\
&= 2^{1} + 2^{2^{0} + 2^{1}} + 2^{2^{2^{1}}} + 2^{2^{0} + 2^{1} + 2^{2^{1}}} + 2^{2^{0} + 2^{2^{0} + 2^{1}}}
\end{aligned}
\end{equation}
Here are some more examples to get familiar,
\begin{align*}
10_2 &= 2^{1} + 2^{2^{0} + 2^{1}}\\
100_2 &= 2^{2^{1}} + 2^{2^{0} + 2^{2^{1}}} + 2^{2^{1} + 2^{2^{1}}}\\
3435_3 &= 2\cdot3^{1} + 3^{3^{1}} + 2\cdot3^{2\cdot3^{0} + 3^{1}} + 3^{2\cdot3^{1}} + 3^{3^{0} + 2\cdot3^{1}}\\
% 1000000000000000001_{10} &= A^{0} + A^{8\cdot A^{0} + A^{1}}
754777787027_{10} &= 7\cdot A^{0} + 2\cdot A^{1} + 7\cdot A^{3} + 8\cdot A^{4} + 7\cdot A^{5} + 7\cdot A^{6} + 7\cdot A^{7} + 7\cdot A^{8} + 4\cdot A^{9} + 5\cdot A^{A^{1}} + 7\cdot A^{A^{0} + A^{1}}
% 1162849439785405935_{10} &= 5\cdot A^{0} + 3\cdot A^{1} + 9\cdot A^{2} + 5\cdot A^{3} + 4\cdot A^{5} + 5\cdot A^{6} + 8\cdot A^{7} + 7\cdot A^{8} + 9\cdot A^{9} + 3\cdot A^{A^{1}} + 4\cdot A^{A^{0} + A^{1}} + 9\cdot A^{2\cdot A^{0} + A^{1}} + 4\cdot A^{3\cdot A^{0} + A^{1}} + 8\cdot A^{4\cdot A^{0} + A^{1}} + 2\cdot A^{5\cdot A^{0} + A^{1}} + 6\cdot A^{6\cdot A^{0} + A^{1}} + A^{7\cdot A^{0} + A^{1}} + A^{8\cdot A^{0} + A^{1}}
\end{align*}
\textbf{Problem Statement:}\\
Output the Hereditary Representation of the input natural number $n$ in base $b$ $(\geq2)$ following the below conventions:

\begin{itemize}
\item Use \texttt{$+, ^*$} to denote addition (add space between operands), multiplication (no space between operands) respectively and \verb!b^{y}! for $b^y$ where $y$ is some expression.
\item The powers of base representation are in increasing order (first $b^0$ then $b^1$ then $b^2$ and so on).
\item Powers are displayed only when their coefficients are $>0$ (non-zero).
\item Coefficients themselves are only displayed when they are $>1$.
\item The exponents from $1$ and $b-1$ must not be simplified further. So, $b$ is represented as \verb!b^{1}! and not as \verb!b^{b^{0}}!.
\item For bases $>10$, use capital alphabets $(A,B,C,\ldots,Z)$ to denote $(10,11,12,\ldots,35)$ respectively.
\end{itemize}
\begin{testcasesFunctionMore}
	{$t$ \hfill(number of test cases, an integer)\\
	$n_1\ b_1\ \quad n_2\ b_2\ \quad \ldots\quad n_t\ b_t$ \hfill($t$ space separated pairs (number, base) for each testcase)}
	{Hereditary Representation of $n_i$ in base $b_i$\hfill(each on a newline)}
	{$1 < n_i \leq 2 \cdot 10^{18}$\\
	$1 < b_i \leq 35$}
	{\texttt{void Hereditary (long long num, int base)} -- prints the required representation}
	{9\\2 2\quad 10 2\quad 100 2\quad 666 3\quad 3435 3\quad3816547290 4\quad3816547290 9\quad 3816547290 35\quad1162849439785405935 10}
	{\texttt{2\^{}\{1\}}\\[0.7em]
\texttt{2\^{}\{1\} + 2\^{}\{2\^{}\{0\} + 2\^{}\{1\}\}}\\[0.7em]
\texttt{2\^{}\{2\^{}\{1\}\} + 2\^{}\{2\^{}\{0\} + 2\^{}\{2\^{}\{1\}\}\} + 2\^{}\{2\^{}\{1\} + 2\^{}\{2\^{}\{1\}\}\}}\\[0.7em]
\texttt{2\textsuperscript{*}3\^{}\{2\} + 2\textsuperscript{*}3\^{}\{3\^{}\{0\} + 3\^{}\{1\}\} + 2\textsuperscript{*}3\^{}\{2\textsuperscript{*}3\^{}\{0\} + 3\^{}\{1\}\}}\\[0.7em]
\texttt{2\textsuperscript{*}3\^{}\{1\} + 3\^{}\{3\^{}\{1\}\} + 2\textsuperscript{*}3\^{}\{2\textsuperscript{*}3\^{}\{0\} + 3\^{}\{1\}\} + 3\^{}\{2\textsuperscript{*}3\^{}\{1\}\} + 3\^{}\{3\^{}\{0\} + 2\textsuperscript{*}3\^{}\{1\}\}}\\[0.7em]
\texttt{2\textsuperscript{*}4\^{}\{0\} + 2\textsuperscript{*}4\^{}\{1\} + 4\^{}\{2\} + 3\textsuperscript{*}4\^{}\{3\} + 3\textsuperscript{*}4\^{}\{4\^{}\{1\}\} + 2\textsuperscript{*}4\^{}\{2\textsuperscript{*}4\^{}\{0\} + 4\^{}\{1\}\} + 3\textsuperscript{*}4\^{}\{3\textsuperscript{*}4\^{}\{0\} + 4\^{}\{1\}\} + 3\textsuperscript{*}4\^{}\{2\textsuperscript{*}4\^{}\{1\}\} + 2\textsuperscript{*}4\^{}\{4\^{}\{0\} + 2\textsuperscript{*}4\^{}\{1\}\} + 3\textsuperscript{*}4\^{}\{2\textsuperscript{*}4\^{}\{0\} + 2\textsuperscript{*}4\^{}\{1\}\} + 4\^{}\{3\textsuperscript{*}4\^{}\{0\} + 2\textsuperscript{*}4\^{}\{1\}\} + 3\textsuperscript{*}4\^{}\{3\textsuperscript{*}4\^{}\{1\}\} + 2\textsuperscript{*}4\^{}\{2\textsuperscript{*}4\^{}\{0\} + 3\textsuperscript{*}4\^{}\{1\}\} + 3\textsuperscript{*}4\^{}\{3\textsuperscript{*}4\^{}\{0\} + 3\textsuperscript{*}4\^{}\{1\}\}}\\[0.7em]
\texttt{2\textsuperscript{*}8\^{}\{0\} + 3\textsuperscript{*}8\^{}\{1\} + 7\textsuperscript{*}8\^{}\{2\} + 8\^{}\{3\} + 6\textsuperscript{*}8\^{}\{4\} + 7\textsuperscript{*}8\^{}\{5\} + 6\textsuperscript{*}8\^{}\{6\} + 3\textsuperscript{*}8\^{}\{7\} + 3\textsuperscript{*}8\^{}\{8\^{}\{1\}\} + 4\textsuperscript{*}8\^{}\{8\^{}\{0\} + 8\^{}\{1\}\} + 3\textsuperscript{*}8\^{}\{2\textsuperscript{*}8\^{}\{0\} + 8\^{}\{1\}\}}\\[0.7em]
\texttt{5\textsuperscript{*}A\^{}\{0\} + 3\textsuperscript{*}A\^{}\{1\} + 9\textsuperscript{*}A\^{}\{2\} + 5\textsuperscript{*}A\^{}\{3\} + 4\textsuperscript{*}A\^{}\{5\} + 5\textsuperscript{*}A\^{}\{6\} + 8\textsuperscript{*}A\^{}\{7\} + 7\textsuperscript{*}A\^{}\{8\} + 9\textsuperscript{*}A\^{}\{9\} + 3\textsuperscript{*}A\^{}\{A\^{}\{1\}\} + 4\textsuperscript{*}A\^{}\{A\^{}\{0\} + A\^{}\{1\}\} + 9\textsuperscript{*}A\^{}\{2\textsuperscript{*}A\^{}\{0\} + A\^{}\{1\}\} + 4\textsuperscript{*}A\^{}\{3\textsuperscript{*}A\^{}\{0\} + A\^{}\{1\}\} + 8\textsuperscript{*}A\^{}\{4\textsuperscript{*}A\^{}\{0\} + A\^{}\{1\}\} + 2\textsuperscript{*}A\^{}\{5\textsuperscript{*}A\^{}\{0\} + A\^{}\{1\}\} + 6\textsuperscript{*}A\^{}\{6\textsuperscript{*}A\^{}\{0\} + A\^{}\{1\}\} + A\^{}\{7\textsuperscript{*}A\^{}\{0\} + A\^{}\{1\}\} + A\^{}\{8\textsuperscript{*}A\^{}\{0\} + A\^{}\{1\}\}}}
	{https://github.com/paramrathour/CS-101/tree/main/Test Cases/Hereditary Representation/Input.txt}
	{https://github.com/paramrathour/CS-101/tree/main/Test Cases/Hereditary Representation/Output.txt}
	{https://github.com/paramrathour/CS-101/tree/main/Starter Codes/Horner's Method.cpp}
\end{testcasesFunctionMore}
\begin{funvideo}
\href{https://youtu.be/uWwUpEY4c8o}{Kill the Mathematical Hydra -- PBS Infinite Series}\\
\href{https://youtu.be/oBOZ2WroiVY}{How Infinity Explains the Finite -- PBS Infinite Series}
\end{funvideo}
\KOMAoptions{paper=A4}
\recalctypearea
\end{document}