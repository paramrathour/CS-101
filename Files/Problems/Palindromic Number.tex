\documentclass[../../Problems]{subfiles}
\begin{document}
\subsection{Palindromic Number}
A non-negative integer is a Palindromic number if it remains the same when it's digits are reversed.

\textbf{Problem Statement:}\\
Determine whether the given integer is a Palindrome for all test cases.
\begin{testcases}
	{$t$ \hfill(number of test cases, an integer)\\
	$n_1\ n_2\ \ldots\ n_t$ \hfill($t$ space separated integers for each testcase)}
	{``yes'' if $n_i$ is a Palindrome else ``no''. \hfill(each test case on a newline)}
	{$0 \leq n_i \leq 10^{9}$}
	{13\\1 7 15 22 196 666 1212 96096 111111 8801088 9256713 40040004 123454321}
	{yes\\yes\\no\\yes\\no\\yes\\no\\no\\yes\\yes\\no\\no\\yes}
	{https://github.com/paramrathour/CS-101/tree/main/Starter Codes/Palindromic Number.cpp}
\end{testcases}
\begin{funvideo}
\href{https://youtu.be/4fE_sXZjxng}{Why 02/02/2020 is the most palindromic date ever. -- Stand-up Maths}\\
\href{https://youtu.be/OKhacWQ2fCs}{Every Number is the Sum of Three Palindromes -- Numberphile}
\end{funvideo}
\end{document}