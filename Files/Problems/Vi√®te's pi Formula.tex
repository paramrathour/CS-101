\documentclass[../../Problems]{subfiles}
\begin{document}
\subsection{Viète's $\pi$ Formula}{\label{pp:vietesformula}}
This problem is a fusion of \ref{pp:wallis} and \ref{pp:ramanujanradical}. It is recommended to solve them before proceeding to this problem.
\begin{equation}
{ {\frac {2}{\pi }}={\frac {\sqrt {2}}{2}}\cdot {\frac {\sqrt {2+{\sqrt {2}}}}{2}}\cdot {\frac {\sqrt {2+{\sqrt {2+{\sqrt {2}}}}}}{2}}\cdots = \prod_{i = 1}^{\infty} \frac{\overbrace{\sqrt{2+\sqrt{\cdots{\sqrt{2+{\sqrt{2+\sqrt{2+0}}}}}}}}^{i\ 2\text{'s}}}{2}}
	\end{equation}
Let's define $\pi_n$ as $n$-th iteration of this infinite nested radical as below
\begin{equation*}
\frac{2}{\pi_n} = \prod_{i = 1}^{n} \frac{\overbrace{\sqrt{2+\sqrt{\cdots{\sqrt{2+{\sqrt{2+\sqrt{2+0}}}}}}}}^{i\ 2\text{'s}}}{2}
\end{equation*}
\textbf{Problem Statement:}\\
Calculate $\pi_n$ for all test cases accurate till $15$ decimal places. See starter code (below) for more details.
\begin{testcases}
	{$t$ \hfill(number of test cases, an integer)\\$n_1\ n_2\ \ldots\ n_t$ \hfill($t$ space separated integers for each testcase)}
	{$\pi_{n_i}$ \hfill(each test case on a newline, accurate till $15$ decimal places)}
	{$1 \leq n_i \leq 50$}
	{8\\1 2 3 5 10 20 30 50}
	{2.828427124746190\\3.061467458920718\\3.121445152258052\\3.140331156954753\\3.141591421511200\\3.141592653588618\\3.141592653589793\\3.141592653589793}
	{https://github.com/paramrathour/CS-101/tree/main/Starter Codes/Viete's pi Formula.cpp}
\end{testcases}
\begin{funvideo}
\href{https://youtu.be/gMlf1ELvRzc}{The Discovery That Transformed Pi -- Veritasium}
\end{funvideo}
\end{document}