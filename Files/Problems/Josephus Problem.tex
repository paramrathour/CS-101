\documentclass[../../Problems]{subfiles}
\begin{document}
\subsection{Josephus Problem}
Suppose there are $n$ terrorists around a circle facing towards the centre. They are numbered $1$ to $n$ along clockwise direction. Initially, terrorist $1$ has the sword. Now, the terrorist with sword kills the $k$\textsuperscript{th} nearest alive terrorist to its left and passes the sword to $(k+1)$\textsuperscript{st} nearest alive terrorist to its left. The process repeats. Basically, every $k$\textsuperscript{th} terrorist is killed until only one survives. Then the last terrorist is killed.
\begin{figure}[H]
    \centering
    \includegraphics[width = 0.15\linewidth]{Josephus.pdf}
    \caption{Example arrangement of 10 terrorists}
    \label{fig:jp}
\end{figure}
\vspace{-1.5em}
For example, in the above arrangement,\\
when $k=1,$ $1$ kills $2$, $3$ kills $4$, $5$ kills $6$, $7$ kills $8$, $9$ kills $10$, $1$ kills $3$, $5$ kills $7$, $9$ kills $1$ and $5$ kills $9$. So, $5$ survives;\\% and is killed last;\\
when $k=2,$ $1$ kills $3$, $4$ kills $6$, $7$ kills $9$, $10$ kills $2$, $4$ kills $7$, $8$ kills $1$, $4$ kills $8$, $10$ kills $5$ and $4$ kills $10$. So, $4$ survives.\\% and is killed last.
\textbf{Problem Statement:}\\
For a given $n, k$ pair, and starting position $1$, print the terrorists in the sequence they are killed.
\begin{testcases}
	{$t$ \hfill(number of test cases, an integer)\\
	$n_1\ k_1\ \quad n_2\ k_2\ \quad \ldots\quad n_t\ k_t$ \hfill($t$ space separated pairs (number of terrorists $n$ and $k$) for each testcase)}
	{Terrorists in the sequence they are killed \hfill(each test case on a newline)}
	{$1 \leq k_i \leq n_i \leq 100$}
	{9\\1 1\quad2 1\quad4 1\quad4 2\quad8 1\quad8 3\quad10 2\quad16 7\quad50 25}
	{1\\2 1\\2 4 3 1\\3 2 4 1\\2 4 6 8 3 7 5 1\\4 8 5 2 1 3 7 6\\3 6 9 2 7 1 8 5 10 4\\8 16 9 2 12 6 3 15 14 1 5 11 10 4 13 7\\26 2 29 6 34 12 41 20 50 32 14 46 30 15 49 37 23 11 3 43 36 28 24 21 19 22 27 35 42 1 10 33 4 25 7 44 38 31 40 5 18 16 39 9 17 45 48 13 8 47}
	{https://github.com/paramrathour/CS-101/tree/main/Starter Codes/Josephus Problem.cpp}
\end{testcases}
\begin{note}
	Verify your program on even more testcases from \href{https://cses.fi/problemset/task/2163/}{here}.
\end{note}
\begin{funvideo}
	\href{https://youtu.be/uCsD3ZGzMgE}{The Josephus Problem -- Numberphile}
\end{funvideo}
\end{document}