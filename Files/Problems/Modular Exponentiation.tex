\documentclass[../../Problems]{subfiles}
\begin{document}
\KOMAoptions{paper=A3}
\recalctypearea
\subsection{Modular Exponentiation}
Consider the problem of calculating $x^y\Mod k$ (i.e. the remainder when $x^y$ is divided by $k$).\\
A naive approach is to keep multiplying by $x$ (and take$\Mod k$) until we reach $x^y$.\footnote{this works because $(a\cdot b)\Mod m =  ((a\Mod m)\cdot(b\Mod m))\Mod m$}
\begin{equation*}
x \Mod k \rightarrow x^2 \Mod k \rightarrow x^3 \Mod k \rightarrow x^4 \Mod k \rightarrow \cdots \rightarrow x^y \Mod k
\end{equation*}
We can use a much faster method which involves \emph{repeated squaring} of $x \Mod k$
\begin{equation}
x \Mod k \rightarrow x^2 \Mod k \rightarrow x^4 \Mod k \rightarrow x^8 \Mod k \rightarrow \cdots \rightarrow x^{2^{\lfloor\log y\rfloor}} \Mod k
\end{equation}
The idea is to multiply some of the above numbers and get $x^y \Mod k$.\\
This is achieved by choosing all powers that have $1$ in binary representation of $y$.\\
For example,
\begin{equation*}
x^{25} = x ^ {11001_2} = x ^ {10000_2} \cdot x ^ {1000_2} \cdot x ^ {1_2} = x ^ {16} \cdot x ^ 8 \cdot x ^ 1
\end{equation*}
which gives,
\begin{equation*}
x^{25} \Mod k  = ((x ^ {16} \Mod k)\cdot (x ^ 8\Mod k)  \cdot (x ^ 1\Mod k)) \Mod k
\end{equation*}
\begin{enumerate}[label=(\alph*)]
\item 
\textbf{Problem Statement:}\\
Calculate $x^y\Mod k$ using the above method for $n$ $(x,y,k)$ triples. Take $k=10^9+7$. \href{https://www.geeksforgeeks.org/modulo-1097-1000000007/}{why this number?}
\begin{testcasesFunction}
	{$t$ \hfill(number of test cases, an integer)\\
	$x_1\ y_1\ \quad x_2\ y_2\ \quad \ldots\quad x_t\ y_t$ \hfill($t$ space separated integer pairs for each testcase)}
	{$x_i^{y_i} \Mod k$ \hfill(each test case on a newline)}
	{$1 < x_i,\ y_i \leq 10^{9}$}
	{\texttt{int mod\_exp(int x, int y, int k)} -- returns $x^y \Mod k$}
	{5\\3 4\quad2 8\quad123 123\quad129612095 411099530\quad241615980 487174929}
	{81\\256\\921450052\\276067146\\838400234}
	{https://github.com/paramrathour/CS-101/tree/main/Starter Codes/Modular Exponentiation I.cpp}
\end{testcasesFunction}
\begin{note}
	Before proceeding to next task, verify your program on more testcases from \href{https://cses.fi/problemset/task/1095}{here}.
\end{note}
\item 
\textbf{Problem Statement:}\\
Calculate $x^{y^z}\Mod k$ using the above method for $n$ $(x,y,z,k)$ fourples. Again, take $k=10^9+7$.
\begin{testcasesFunction}
	{$t$ \hfill(number of test cases, an integer)\\
	$x_1\ y_1\ z_1\ \quad x_2\ y_2\ z_2\ \quad \ldots\quad x_t\ y_t z_t\ $ \hfill($t$ space separated triples for each testcase)}
	{$x_i^{y_{i}^{z_i}} \Mod k$ \hfill(each test case on a newline)}
	{$1 < x_i,\ y_i,\ z_i \leq 10^{9}$}
	{\texttt{int mod\_super\_exp(int x, int y, int z, int k)} -- returns $x^{y^z} \Mod k$}
	{5\\3 7 1\quad15 2 2\quad3 4 5\quad427077162 725488735 969284582\quad690776228 346821890 923815306}
	{2187\\50625\\763327764\\464425025\\534369328}
	{https://github.com/paramrathour/CS-101/tree/main/Starter Codes/Modular Exponentiation II.cpp}
\end{testcasesFunction}
\begin{note}
	Verify your program on more testcases from \href{https://cses.fi/problemset/task/1712}{here}.
\end{note}
\end{enumerate}
\begin{funvideo}
\href{https://youtu.be/cbGB__V8MNk}{Square \& Multiply Algorithm - Computerphile}
\end{funvideo}
\KOMAoptions{paper=A4}
\recalctypearea
\end{document}