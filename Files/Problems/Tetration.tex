\documentclass[../../Problems]{subfiles}
\begin{document}
\subsection{Tetration}{\label{pp:tetration}}
Problem \ref{pp:harmonic} is about repeated additions whereas \ref{pp:wallis} is about repeated multiplication. Guess what this problem is about? Yes! It's repeated exponentiation. Tetration, the next \href{https://en.wikipedia.org/wiki/Hyperoperation}{hyperoperation} after exponentiation defined as:%. It is defined as repeated exponentiation
\begin{equation}
{^{n}a}=\underbrace {a^{a^{\cdot ^{\cdot ^{a}}}}} _{n} \quad\text{repeated exponentiation}
\end{equation}
A generalised way to denote hyperoperations is using the \href{https://en.wikipedia.org/wiki/Knuth's_up-arrow_notation}{Knuth's up-arrow notation} and the successor function:
\begin{align}
\text{Succession }\quad &      a\uparrow ^{-2}b  &&    &&                                                                                                                            &=&\ a + 1\\
\text{Addition }\quad &        a\uparrow ^{-1}b &=&\ a\underbrace{\uparrow^{-2}\cdots\uparrow^{-2}}_{b\ \text{times}} b &=&\ a + \underbrace {1+ 1+ \cdots + 1} _{b\ \text{times}}       &=&\ a+ b\\
\text{Multiplication }\quad &  a\uparrow ^{0}b &=&\ a\underbrace{\uparrow^{-1}\cdots\uparrow^{-1}}_{b\ \text{times}} b &=&\ \underbrace {a+ a+ \cdots + a} _{b\ \text{times}}            &=&\ a\times b\\
\text{Exponentiation }\quad &  a\uparrow ^{1}b &=&\ a\underbrace{\uparrow^0\cdots\uparrow^0}_{b\ \text{times}} b &=&\ \underbrace {a\times a\times \cdots \times a} _{b\ \text{times}}   &=&\ a^{b}\\
\text{Tetration }\quad &       a\uparrow ^{2}b &=&\ a\underbrace{\uparrow^1\cdots\uparrow^1}_{b\ \text{times}} b &=&\ \underbrace {a^{a^{\cdot ^{\cdot ^{a}}}}} _{b\ \text{times}}       &=&\ {^{b}}a\\
\text{Pentation }\quad &       a\uparrow ^{3}b &=&\ a\underbrace{\uparrow^2\cdots\uparrow^2}_{b\ \text{times}} b &=&\ \underbrace {{^{{^{{^{{^{a}}\cdot}}\cdot}}a}}a}_{b\ \text{times}}  &&\text{and so on}
\end{align}
\textbf{Problem Statement:}\\
Calculate ${^{n}a}$ for all test cases accurate till $10$ decimal places. See starter code (below) for more details.
\begin{testcases}
	{$t$ \hfill(number of test cases, an integer)\\
	$a_1\ n_1\quad a_2\ n_2\quad \ldots\quad a_t\ n_t$ \hfill($t$ space separated pairs for each testcase)}
	{${^{n}a}$ \hfill(each test case on a newline, accurate till $10$ decimal places)}
	{$0.05 \leq a_i \leq 3$\hfill(a double)\\
	$1 \leq n_i \leq 1000$ \hfill(an integer)\\${^{n_i}a_i}$ is within the range of \texttt{double} data type}
	{10\\1 1\quad1 2\quad2 1\quad2 2\quad2 3\quad3 2\quad3 3\quad1.41421356237 20\quad0.06598803584 1000 \quad	1.44466786101 1000}
	{1.0000000000\\1.0000000000\\2.0000000000\\4.0000000000\\16.0000000000\\27.0000000000\\7625597484987.0000000000\\1.9995856229\\0.3968311347\\2.7128728643}
	{https://github.com/paramrathour/CS-101/tree/main/Starter Codes/Tetration.cpp}
\end{testcases}
\begin{funvideo}
\href{https://youtu.be/txajrEOTkuY}{Graham's Number Escalates Quickly -- Numberphile} (uses a different arrow-notation where $a\uparrow b=a\uparrow ^{0}b$)\\
\href{https://youtu.be/DmP3sFIZ0XE}{``Prove'' 4 = 2 Using Infinite Exponents. Can You Spot The Mistake? -- Mind Your Decisions}
\end{funvideo}
\end{document}