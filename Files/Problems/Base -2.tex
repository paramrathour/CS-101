\documentclass[../../Problems]{subfiles}
\begin{document}
\subsection{Base --2}
By using $-2$ as the base, both positive and negative integers can be expressed without an explicit sign or other irregularity. Just like positive integral bases, any base $-2$ number can be represented as follows:
\begin{equation}
(a_n\ldots a_2a_1a_0)_{(-2)} = a_n(-2)^n+\cdots+a_2(-2)^2+a_1(-2)^1+a_0(-2)^0 \quad\text{where $a_i$ is either 0 or 1}
\end{equation}
To find base $-2$ representation of $n$, we repeatedly divide by $-2$ until the quotient becomes 0 and the remainders generated (which are either 0 or 1) will be the digits of base $-2$ representation.
\begin{equation*}
n = \text{Quotient}\times(-2) + \text{Reminder} \quad\rightarrow\quad \text{Quotient} = \text{Quotient}_{\text{new}}\times(-2) + \text{Reminder}_{\text{new}}
\end{equation*}
For $-3$, the process it as shown below,
\begin{equation*}
\begin{aligned}
-3&= 2\times(-2) + {1} &\quad\rightarrow\quad a_0 = 1\\
2&= -1\times(-2) + {0} &\quad\rightarrow\quad a_1 = 0\\
-1&= 1\times(-2) + {1} &\quad\rightarrow\quad a_2 = 1\\
1&= 0\times(-2) + {1} &\quad\rightarrow\quad a_3 = 1
\end{aligned}
\end{equation*}
Hence $(-3)_{10} = (1101)_{(-2)}$.
\begin{note}
C++'s \% operator may give negative values when the dividend or divisor is negative.\\
For example, $(-1)\%(2) = (-1)\%(-2) = -1 \neq 1$.
\end{note}

\textbf{Problem Statement:}\\
Convert the given decimal number into base $-2$ for all test cases.
\begin{testcasesMore}
	{$t$ \hfill(number of test cases, an integer)\\
	$n_1\ n_2\ \ldots\ n_t$ \hfill($t$ space seperated integers for each testcase)}
	{Converted base $-2$ number \hfill(each test case on a newline)}
	{$-200 \leq n_i \leq 200$}
	{10\\-4 -3 -2 -1 0 1 2 3 4 100}
	{1100\\1101\\10\\11\\0\\1\\110\\111\\100\\110100100}
	{https://github.com/paramrathour/CS-101/tree/main/Test Cases/Base -2/Input.txt}
	{https://github.com/paramrathour/CS-101/tree/main/Test Cases/Base -2/Output.txt}
	{https://github.com/paramrathour/CS-101/tree/main/Starter Codes/Base -2.cpp}
\end{testcasesMore}
\begin{funvideo}
\href{https://youtu.be/U6xJfP7-HCc}{Base 12 -- Numberphile}
\end{funvideo}
\end{document}