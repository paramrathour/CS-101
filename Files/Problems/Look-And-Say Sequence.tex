\documentclass[../../Problems]{subfiles}
\begin{document}
\subsection{Look-And-Say Sequence}
As the name suggests, the look-and-say sequence is generated by the \emph{reading} of the digits of the previous sequence.\\% in a \emph{look-and-say manner}.\\
For example, starting with the sequence \textbf{1}.
\begin{itemize}
\item \textbf{1} is read off as ``one 1'' or \textbf{11}.
\item \textbf{11} is read off as ``two 1s'' or \textbf{21}.
\item \textbf{21} is read off as ``one 2, one 1'' or \textbf{1211}.
\item \textbf{1211} is read off as ``one 1, one 2, two 1s'' or \textbf{111221}.
\item \textbf{111221} is read off as ``three 1s, two 2s, one 1'' or \textbf{312211} and so on.
\end{itemize}
\textbf{Problem Statement:}\\
Generate the first $n$ iterations of the look-and-say sequence.
\begin{testcasesMore}
	{$n$ \hfill(a single integer)}
	{First $n$ iterations of the look-and-say sequence\hfill(each iteration on a newline)}
	{$1 \leq n \leq 40$}
	{15}
	{1\\[0.5em]11\\[0.5em]21\\[0.5em]1211\\[0.5em]111221\\[0.5em]312211\\[0.5em]13112221\\[0.5em]1113213211\\[0.5em]31131211131221\\[0.5em]13211311123113112211\\[0.5em]11131221133112132113212221\\[0.5em]3113112221232112111312211312113211\\[0.5em]1321132132111213122112311311222113111221131221\\[0.5em]11131221131211131231121113112221121321132132211331222113112211\\[0.5em]311311222113111231131112132112311321322112111312211312111322212311322113212221}
	{https://github.com/paramrathour/CS-101/tree/main/Test Cases/Look-And-Say Sequence/Input.txt}
	{https://github.com/paramrathour/CS-101/tree/main/Test Cases/Look-And-Say Sequence/Output.txt}
	{https://github.com/paramrathour/CS-101/tree/main/Starter Codes/Look-And-Say Sequence.cpp}
\end{testcasesMore}
\begin{funvideo}
	\href{https://youtu.be/ea7lJkEhytA}{Look-and-Say Numbers (feat John Conway) -- Numberphile}\\
	\href{https://youtu.be/EGoRJePORHs}{Can you trust an elegant conjecture? -- Stand-up Maths}
\end{funvideo}
\end{document}