\documentclass[../../Problems]{subfiles}
\begin{document}
\subsection{Ackermann Function}
Ackermann Function is defined as follows
\begin{equation}
	\begin{aligned}
		\op{A}(0,n)&=n+1\\
		\op{A}(m,0)&=\op{A}(m-1,1)\\
		\op{A}(m,n)&=\op{A}(m-1,\op{A}(m,n-1))
	\end{aligned} \quad\text{or in terms of \hyperref[pp:tetration]{arrow-notation}}\quad
		\begin{aligned}
		\op{A}(m,n) =
			\begin{cases}n+1&m=0\\
			2\uparrow ^{m-2}(n+3)-3&m>0
			\end{cases}
	\end{aligned}
\end{equation}
\textbf{Problem Statement:}\\
Calculate $\op{A}(m,n)$ (given $m,n$) for all test cases.
\begin{testcasesFunction}
	{$t$ \hfill(number of test cases, an integer)\\
	$m_1\ n_1\ \quad m_2\ n_2\ \quad \ldots\quad m_t\ n_t$ \hfill($t$ space separated integer pairs for each testcase)}
	{$\op{A}(m_i,n_i)$\hfill(each on a newline)}
	{$m_i,n_i$ are postive integers such that $\op{A}(m_i,n_i)$ is within the range of int}
	{\texttt{int A(int m, int n)} -- returns $A(m,n)$}
	{10\\0 0\quad0 5\quad1 0\quad1 3\quad2 4\quad3 1\quad3 3\quad3 9\quad4 0\quad4 1}
	{1\\6\\2\\5\\11\\13\\61\\4093\\13\\65533}
	{https://github.com/paramrathour/CS-101/tree/main/Starter Codes/Ackermann Function.cpp}
\end{testcasesFunction}
\begin{noteI}
	Was your program able to compute the last output? Why not? How to fix this?
\end{noteI}
\begin{funvideo}
\href{https://youtu.be/i7sm9dzFtEI}{The Most Difficult Program to Compute? -- Computerphile}
\end{funvideo}
\end{document}