\documentclass[../../Problems]{subfiles}
\begin{document}
\subsection{Partitions}
A partition of a natural number $n$ is a way of decomposing $n$ as sum of natural numbers $\leq n$.\\
For example, their are $5$ partitions of $4$ given by $\{4, 3+1, 2+2, 2+1+1, 1+1+1+1\}$.\\
Let use denote the number of partitions of $n$ by $\op{P}(n)$.\\
Now, we move to a seemingly unrelated theorem.
\begin{theorem}[Pentagonal Number Theorem]
	PNT relates the product and series representations of the \href{https://en.wikipedia.org/wiki
	/Euler_function}{Euler function}
	\begin{equation}{\label{eq:pnt}}
		\prod_{n=1}^{\infty}\left(1-x^{n}\right)=\sum_{k=-\infty }^{\infty }\left(-1\right)^{k}x^{k\left(3k-1\right)/2}=1+\sum _{k=1}^{\infty }(-1)^{k}\left(x^{k(3k+1)/2}+x^{k(3k-1)/2}\right)
	\end{equation}
	In other words,
	\begin{equation*}
		(1-x)(1-x^{2})(1-x^{3})\cdots =1-x-x^{2}+x^{5}+x^{7}-x^{12}-x^{15}+x^{22}+x^{26}-\cdots
	\end{equation*}
	The exponents $1, 2, 5, 7, 12,\ldots$ on the right hand side are called (generalized) pentagonal numbers (\href{https://oeis.org/A001318}{A001318}).\\
	They are given by the formula $p_k = k(3k - 1)/2$ for $k = 1, -1, 2, -2, 3,-3,\ldots$
\end{theorem}
Equation \ref{eq:pnt} implies a recurrence relation for calculating $\op{P}(n)$ given by
\begin{equation}
	\op{P}(n)=\op{P}(n-1)+\op{P}(n-2)-\op{P}(n-5)-\op{P}(n-7)+\cdots = \sum_{k\neq 0}(-1)^{k-1}\op{P}(n-p_{k})
\end{equation}
\textbf{Problem Statement:}\\
Calculate $\op{P}(n)$ for all test cases using \ref{eq:pnt} or otherwise :).
\begin{testcasesFunction}
	{$t$ \hfill(number of test cases, an integer)\\
	$n_1\ n_2\ \ldots\ n_t$ \hfill($t$ space separated integers for each testcase)}
	{$\op{P}(n_i)$ \hfill(each test case on a newline)}
	{$1 \leq n_i \leq 40$}
	{\texttt{int P(int n)} -- returns $\op{P}(n)$ }%\hfill(try solving with and without using the given extra parameter $k$ :D)}
	{9\\1 2 3 4 5 10 20 30 40}
	{1\\2\\3\\5\\7\\42\\627\\5604\\37338}
	{https://github.com/paramrathour/CS-101/tree/main/Starter Codes/Partitions.cpp}
\end{testcasesFunction}
\begin{funvideo}
\href{https://youtu.be/NjCIq58rZ8I}{Partitions -- Numberphile}\\
\href{https://youtu.be/iJ8pnCO0nTY}{The hardest What comes next (Euler's pentagonal formula) -- Mathologer}
\end{funvideo}
\end{document}