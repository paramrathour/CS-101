\documentclass[../../Problems]{subfiles}
\begin{document}
\subsection{Shoelace Formula}
Shoelace Formula determines the area of a \href{https://en.wikipedia.org/wiki/Simple_polygon}{simple polygon} whose vertices are given by Cartesian coordinates.
\begin{equation}{\label{eq:shoelace}}
A = \frac{{\begin{vmatrix}x_{1}&x_{2}&x_{3}\quad\cdots &x_{n}&x_{1}\\y_{1}&y_{2}&y_{3}\quad \cdots &y_{n}&y_{1}\end{vmatrix}}}{2}
\end{equation}
which can be simplfied as
\begin{equation*}
A = \frac{{\begin{vmatrix}x_{1}&x_{2}\\y_{1}&y_{2}\end{vmatrix}}+{\begin{vmatrix}x_{2}&x_{3}\\y_{2}&y_{3}\end{vmatrix}}+\cdots +{\begin{vmatrix}x_{n}&x_{1}\\y_{n}&y_{1}\end{vmatrix}}}{2}\quad\text{where}\quad
\begin{vmatrix}x_{i}&x_{j}\\y_{i}&y_{j}\end{vmatrix} = x_i\cdot y_j - x_j\cdot y_i
\end{equation*}
\textbf{Problem Statement:}\\
Calculate the area of a given $n$-sided polygon for all test cases accurate till $1$ decimal place.
\begin{testcasesMore}
	{$t$ \hfill(number of test cases, an integer)\\
	$n_i\qquad x_1\ y_1\quad x_2\ y_2\quad \cdots\quad  x_n\ y_n\quad$ \hfill($2n_i+1$ space separated integers for each testcase)}
	{$A_{i}$ \hfill(each test case on a newline, accurate till $1$ decimal places)}
	{$3 \leq n_i \leq 1000\\ -10^5 \leq x_i, y_i \leq 10^5$\\
	The given polygon is simple.}
	{6\\3\qquad0 1\quad2 3\quad4 7\\3\qquad1 1\quad5 9\quad3 5\\3\qquad3 4\quad1 1\quad4 1\\4\qquad-2 4\quad-2 1\quad3 -3\quad4 4\\8\qquad458 695\quad 621 483\quad 877 469\quad 1035 636\quad 1061 825\quad 875 1023\quad 645 1033\quad 485 853\quad\\10\qquad 443 861\ 470 506\ 754 432\ 910 446\ 952 485\ 1036  595\ 1101 721\ 1045 954\ 947 1009\ 712 1095\ }
	{2.0\\0.0\\4.5\\28.5\\255931.0\\325573.5}
	{https://github.com/paramrathour/CS-101/tree/main/Test Cases/Shoelace Formula/Input.txt}
	{https://github.com/paramrathour/CS-101/tree/main/Test Cases/Shoelace Formula/Output.txt}
	{https://github.com/paramrathour/CS-101/tree/main/Starter Codes/Shoelace Formula.cpp}
\end{testcasesMore}
\begin{funvideo}
\href{https://youtu.be/0KjG8Pg6LGk}{Gauss's magic shoelace area formula and its calculus companion -- Mathologer}
\end{funvideo}
\end{document}